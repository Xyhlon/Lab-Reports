%in Grundlagen
Operationsverstärker (kurz 'OPV oder 'OpAmp') dienen der Verstärkung von
Gleichspannungen. Sie besitzen einen nicht-invertierenden, der meist mit einem
Plus gekennzeichnet ist, und einen invertierenden Eingang, der häufigst mit
einem Minus dargestellt wird. Zu beachten ist, dass die Verstärkung auf die
Differenzspannung der beiden Eingänge wirkt. Je zwei zusätzliche Anschlüsse
finden sich für die positive und negative Betriebsspannung und für den
Offsetabgleich, damit bei keiner Eingangsspannung auch keine Ausgangsspannung
auftritt - dieser wird also in einer externen Schaltung durchgeführt.
%In \autoref{fig:pin_anschl} sind die Pins eines Operationsverstärkers, wie er
%auch in der Laborübung verwendet wurde, zu sehen.


Es gibt vier grundlegende Arten der Verwendung von Operationsverstärkern,
darunter der nicht-invertierende Betrieb, bei dem das Eingangssignal nur auf
den nicht-invertierenden Kanal gelegt wird und der invertierende auf Masse
gelegt wird. Analog funktioniert der invertierende Modus, bei dem das Signal
nun an den invertierenden Eingang gelegt wird, wodurch die Ausgangsspannung
zusätzlich zur Verstärkung noch zum Eingangssignal invertiert wird. Beim
Differenzbetrieb werden an beide Eingänge Signale angelegt und die
Differenzspannung verstärkt. Im Falle des Gleichtaktbetriebs liegt das gleiche
Eingangssignal an den beiden Eingängen an, wodurch es theoretisch keine
Differenzspannung und Verstärkung geben sollte - in der Realität resultiert
allerdings eine Verstärkung, die als Gleichtkatkverstärkung bezeichnet wird.

Da der Operationsverstärker ohne zusätzliche Verkopplung sehr stark
frequenzabhängig ist und nur eine geringe Bandbreite gewünscht verstärkt, wird
eine Gegenkopplung vom Ausgang zum Eingang durchgeführt, wodurch die
Verstärkung zwar abnimmt, die Bandbreite jedoch stark vergrößert wird. Die
Bandbreite wird wie gewohnt durch die Grenzfrequenz chraktersisiert, bei
welcher die Verstärkung noch \SI{70}{\%} der maximalen beträgt.

Die resultierende Verstärkung lässt sich gemäß \autoref{eq:ver} als Verhätlnis
der Eingangs- $U_e$ zur Ausgangsspannung $U_a$ berechnen.
%insert eq ---


%in Durchführung
Zu beachten ist, dass jeder nicht belegte Pin auf Massenpotential gelegt wird.
