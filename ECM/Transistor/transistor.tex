%! TeX program = lualatex
%---------------------------ALLGEMEINE IMPORTS-------------------------------------
\documentclass[12pt,english,ngerman]{scrartcl}

\input{./input/shared_preamble.tex}

    % Kopfzeile
\ihead{SS22\\06.04.2022}
\chead{\textsc{Hinterleitner} Michael - 12002411 \\ \textsc{Philipp} Maximilian - 11839611}
\ohead{LU ECM-\\ Transistor}
    % Fußzeile
%--------------------------------------AB HIER DOKUMENT---------------------------------------------
\begin{document}
\includepdf{deckblatt1.pdf}
\tableofcontents
\newpage


%\section{Aufgabenstellung}\label{sec:Aufgabenstellung}

% Die nachfolgende Aufgabenstellung wurde von den Laborbetreuern bereitgestellt
% und beinhaltet sowohl Angaben zur Vorbereitung als auch zur praktischen
% Durchführung der Übung:

% zu 1: Aufgabenstellung Das vor der Übung verteilte Aufgabenblatt.
 \includepdf[
     pages=-,  % all pages
     addtotoc={
         1, section, 1, Aufgabenstellung, sec:Aufgabenstellung
     },
     addtolist={
        4, figure, {Emitterschaltung mit Gleichstromgegenkopplung}, fig:angabe_abb1,
        4, figure, {Ersatzschaltbild Eingangs- und Ausgangswiderstand}, fig:angabe_abb2,
        4, figure, {Ersatzschaltbild Koppelkondensatoren}, fig:angabe_abb3
     }
 ]{angabe.pdf}

% zu 2: Vorbereitung Es sind beide Vorbereitungen dem Protokoll beizufügen.
%\section{Vorbereitung}\label{sec:Vorbereitung}
%Die folgende Vorbereitung wurde vor der Laborübung 
\includepdf[pages=-,
     addtotoc={
         1, section, 2, Vorbereitung, sec:Vorbereitung
     }]{./figures/Transistor1.pdf}
\includepdf[pages=-]{./figures/Transistor2.pdf}


% zu 3: Grundlagen In den Grundlagen sollen die später verwendeten Formeln
% stehen und kurz erklärt werden, dabei ist es nicht notwendig Formeln
% herzuleiten. Quellenangaben sind an dieser Stelle von Vorteil, weil Sie so
% schnell die betreffenden Stellen in Unterlagen finden. In den Rechnungen
% werden grundlegende Annahmen skizziert und begründet und dann mit diesen
% Annahmen, die für die Schaltungen notwendigen Werte berechnet. Dabei kann
% auch gleich auf die später wirklich verwendeten Werte Bezug genommen werden -
% wir verwenden bei den Widerständen zum Beispiel von den Normwert-Reihen die
% E12 und/oder E24 Serie (nach DIN 41426 bzw. IEC 63).
\section{Grundlagen}\label{sec:Grundlagen}

Bipolartransistoren sind Halbleiterbauelemente mit zwei pn-Halbleiterübergängen
(entweder npn oder pnp), bei denen gegensätzlich zu den Feldeffekttransistoren
beide Arten von Ladungsträgern, Elektronen und Löcher/Defektelektronen, am
Stromfluss beteiligt sind. Für Schaltungen mit Bipolartransistoren, die in der
Elektronik zur Verstärkung respektive Schaltung verwendet werden, wird zwischen
Emitter-, Basis- und Kollektorschaltung differenziert. Die
Schaltungsbezeichnungen beruhen auf dem Anschluss, der als Bezug für Eingang
und Ausgang dient. Diese Schaltungsarten sind in 
\autoref{fig:schaltungsarten} dargestellt; zu beachten ist, dass der Pfeil am
Emitter des Transistors in die technische Stromrichtung zeigt. 

\begin{figure}[H]
    \centering
    \includegraphics[width=8cm, height=8cm,keepaspectratio]{./figures/schaltungsarten.png}
    \caption{Darstellung der 3 Schaltungsarten mit Bipolartransistor \cite{tietze}}
    \label{fig:schaltungsarten}
\end{figure}

In dieser Laborübung sind ausschließlich Emitterschaltungen von Relevanz,
insbesondere jene mit Stromgegenkopplung, wodurch die Temperaturabhängigkeit
der Schaltung kompensiert wird. Dies ist deswegen von hoher Relevanz, da
Transistorschaltungen stets um einen bestimmten Arbeitspunkt betrieben werden
(sollten). Dieser Arbeitspunkt wird über den Kollektorstrom $I_C$, Basisstrom
$I_B$, der Kollektor-Emitterspannung $U_{CE}$ beziehungsweise
Basis-Emitterspannung $U_{BE}$ festgelegt. Ein Anstieg der Temperatur würde
diesen wiederum, aufgrund der Temperaturabhängigkeit von Halbleitern
(pn-Übergang), verschieben. Dadurch wird eine Zunahme des Basisstroms $I_B$ und
infolge des Kollektorstroms $I_C$ sowie eine Abnahme des Kollektorpotentials
$V_C$ bedingt. Um dies zu kompensieren, wird im Rahmen der Stromgegenkopplung
ein Emitterwiderstands $R_E$ implementiert. Dieser führt aufgrund des nun
höheren Emitterstroms $I_E$, der sich gemäß der Kirchhoff'schen Knotenregel aus
der Summe der Teilströme $I_C$ und $I_B$ (der allerdings vernachlässigt werden
kann) ergibt, zu einer größeren Spannung $U_{RE}$, die am Emitterwiderstand
abfällt. Dadurch nimmt die Basis-Emitterspannung $U_{BE}$ ab und der Basisstrom
wird geringer, genauso wie folglich der Kollektor- und Emitterstrom, was der
ursprünglichen Erhöhung entgegenwirkt. \cite{tietze}
\newline

Zur Berechnung der Ströme am Steckbrett (Kapitel \nameref{sec:Auswertung}), des
Basis- $I_B$ und Kollektorstroms $I_C$, nachdem die an den Vorwiderständen
$R_1$ und $R_2$ abfallenden Spannungen $U_{R1}$ respektive $U_{R2}$ gemessen
wurden, wird das Ohmsche Gesetz \autoref{eq:ohm} verwendet. Dabei bezeichnet
wie gewohnt $U$ die Spannung, $I$ den Strom und $R$ den Widerstand als die
Proportionalitätskonstante beziehungsweise für einen nicht-linearen Verlauf $r$
den differentiellen Widerstand an einem Arbeitspunkt. \cite{tietze}

\begin{align}
	R&=\frac{U}{I} \label{eq:ohm}\\
	r&= \der{U}{I} \Bigr|_{\substack{Arbeitspunkt}}
\end{align}

Die (Spannungs-)Verstärkung $V$ der Schaltung ergibt sich aus dem Verhältnis
zwischen Ausgangsspannung und Eingangsspannung gemäß \autoref{eq:verst}.

\begin{equation}
	V=\frac{U_a}{U_e}
	\label{eq:verst}
\end{equation}

% zu 4: Versuchsdurchführung In diesem Punkt wird die Durchführung der
% einzelnen Aufgaben beschrieben. Im Simulationsteil ist die simulierte
% Schaltung mit allen Analyseparametern darzustellen. Im praktischen Teil sind
% die verwendete Geräte sowie die gemessenen Werte der verwendeten Bauteile
% anzugeben. Außerdem sind durchgeführte Funktionsüberprüfungen der Bauteile
% (Dioden, Transistor, etc.) anzuführen. Die Messergebnisse bzw. Oszillogramme
% sind mit Angabe der verwendeten Messgeräte anzugeben. Oszillogramme werden
% vom verwendeten Oszilloskop als Daten auf einen USB-Stick ausgegeben und
% können in das Protokoll aufgenommen werden. Das gleiche gilt für Schaltungen
% bzw. Ergebnissen von Simulationen. Es ist auf eine klare Darstellung der
% Messergebnisse und –auswertung zu achten (Tabellen, geeignete Grafiken). Die
% originalen, während des Versuchs angefertigten Aufzeichnungen sind dem
% Protokoll beizufügen. 
\section{Versuchsdurchführung}\label{sec:Versuchsdurchf}

\subsection{Simulation} \label{sec:Versuchsim}

Zur Simulation der Emitterschaltung mit Gegenstromkopplung wird das Programm
\textit{LTSPICE} verwendet. Der Aufbau erfolgt analog zum skizzierten Schaltplan im
Kapitel \nameref{sec:Vorbereitung}. Auf die (Gerät-)Spezifikationen ist dementsprechend zu achten.

% DONE 1.1 Bauen Sie die Schaltung mit dem Programm PSPICE LT gemäß der berechneten
% DONE Parameter ohne den Überbrückungskondensator CE auf.
\subsubsection{Schaltung ohne Überbrückungskondensator} \label{sec:Versuchohnekond}

Zunächst wird die Schaltung allerdings ohne Überbrückungskondensator, wie sie
in  \autoref{fig:schaltungohnekond} zu sehen ist, verwendet. Nachdem
alle Parameter gemäß den Angaben in der Simulation und insbesondere der
Arbeitspunkt entsprechend dem theoretisch errechneten Wert eingestellt wurden
(ca. \SI{7.5}{\volt}), wurden die Eingangs- und Ausgangsspannung über der Zeit in
einem Plot, durch Messung dieser Größen für einen geeigneten Zeitabschnitt
(siehe  \autoref{fig:schaltungohnekond} im unteren Bildbereich),
dargestellt. 

\begin{figure}[H]
    \centering
    \includegraphics[width=\textwidth, height=6cm,keepaspectratio]{./figures/ohnekond/schaltungmitmessungen20mv.png}
    \caption{Schaltung ohne Überbrückungskondensator bei einem Sinus-Eingangssignal
    mit einer Amplitude von \SI{20}{mV}, einer Frequenz von \SI{1}{k\hertz} und einem
    Innenwiderstand von \SI{600}{\ohm} ; mit der Eingangs- $U_e$ und
    Ausgangsspannung $U_a$, den Eingangs- $Ce$ und Ausgangskondensatoren $Ca$, den
    Vorwiderständen $R1$ und $R2$, dem Kollektorwiderstand $RC$, dem
    Emitterwiderstand $RE$, dem Lastwiderstand $RL$, dem Basispotential $Vb$, dem
    Kollektorpotential $Vc$, dem Transistor \textit{BC107B} und der
    Betriebsspannung $UB$. Genauere Spezifikationen können dem Schaltbild entnommen
    werden.}
    \label{fig:schaltungohnekond}
\end{figure}

% FIXME(MAX) 1.2 Stellen Sie eine Eingangs-Sinusspannung von 1 kHz mit einer Amplitude innerhalb der
% FIXME(MAX) Übersteuerungstoleranzen ein, erzeugen Sie ein Simulation Profil (Time Domain) und nehmen Sie
% FIXME(MAX) jeweils die Eingangsspannung und die Ausgangsspannung in einem Plot über die Zeit auf. Lassen Sie
% FIXME(MAX) sich auch die Spannungen und Ströme im Schaltbild anzeigen 
% um den Arbeitspunkt diskutieren zu
% können. Berechnen Sie daraus die simulierte Verstärkung und diskutieren Sie die beiden Diagramme
% und ihren Zusammenhang.

\paragraph{Normalbetrieb}

In \autoref{fig:verlaufohnekond} respektive \autoref{fig:sim_ohne_normal_ausgang} ist der zeitliche Verlauf von Eingangs-
und Ausgangsspannung der Emitterschaltung ohne Überbrückungskondensator im Normalbetrieb
dargestellt.

\begin{figure}[H]
    \centering
    \includegraphics[width=\linewidth, height=7cm]{./figures/ohnekond/eingangssignal20mv.png}
    \caption{Zeitlicher Verlauf der Eingangsspannung $V(u_e)$ 
    der Emitterschaltung ohne Überbrückungskondensator bei einer Amplitude von
  \SI{20}{mV} des Eingangssignals}
    \label{fig:verlaufohnekond}
\end{figure}

\begin{figure}[H]
    \centering
    \includegraphics[width=\linewidth, height=7cm]{./figures/ohnekond/ausgangssignal20mv.png}
    \caption{Zeitlicher Verlauf der Ausgangsspannung $V(u_a)$
    der Emitterschaltung ohne Überbrückungskondensator bei einer Amplitude von
  \SI{20}{mV} des Eingangssignals}
    \label{fig:sim_ohne_normal_ausgang}
\end{figure}

% FIXME(MAX) 1.3 Testen Sie wie hoch die maximale Eingangspannung werden darf bis der Transistor in der Simulation
% FIXME(MAX) übersteuert. Übersteuern Sie ihn anschließend und nehmen Sie wieder die Eingangspannung sowie
% FIXME(MAX) die Ausgangsspannung nach der Zeit auf 
% und diskutieren Sie anhand dieses Plots die auftretenden
% Verzerrungen.

\paragraph{Übersteuerungsbetrieb}

Um die Übersteuerungsgrenze zu finden, wurde ein Parameter-Sweep durchgeführt,
wobei die Amplitude der Eingangsspannung variiert wurde. In 
\autoref{fig:sim_ohne_paramsweep_ausgang} ist der Verlauf der Ausgangsspannung
in Abhängigkeit der Eingangsspannung dargestellt. Dadurch konnte die
Übersteuerungsgrenze visuell bestimmt werden.

\begin{figure}[H]
    \centering
    \includegraphics[width=\linewidth, height=7cm]{./figures/ohnekond/parametersweepuebersteuerung.png}
    \caption{Zeitlicher Verlauf der Ausgangsspannung $V(u_a)$
    der Emitterschaltung ohne Überbrückungskondensator mit einem Parameter Sweep der Amplitude des Eingangssignals. 
    Folgende SPICE directives wurden für die Simulation verwendet
    \texttt{.tran 0.004 0.007 0.004 0.00001} und \texttt{.step param X 0mV 350mV 20mV}}
    \label{fig:sim_ohne_paramsweep_ausgang}
\end{figure}

% FIXME(MAX) 1.4 Erstellen Sie einen DC Sweep in Abhängigkeit der Temperatur und zeigen Sie die Änderung des
% FIXME(MAX) Kollektorpotentials. 
% Diskutieren Sie die Konsequenzen einer Temperaturerhöhung.
\paragraph{DC Temperatur Sweep}
Zur Darstellung der Temperaturabhängigkeit wurde ein DC-Sweep durchgeführt. In \autoref{fig:sim_dc_temp_sweep_ohne} ist das Kollektorpotential in Abhängigkeit der Temperatur zu sehen.
\begin{figure}[H]
  \centering
    \includegraphics[width=\linewidth, height=7cm]{./figures/ohnekond/tempsweepdclinear.png }
    \caption[Simulierter DC Temperatur Sweep ohne
    Überbrückungskondensator]{Simulierter DC Temperatur Sweep, welcher die
      Temperaturabhängigkeit des Kollektorpotentials darstellt. Auf der
      Abszisse befindet sich die Temperatur des Transistors und auf der Ordinate
      das Kollektorpotential $VC$. Diese Simulation wurde für den
      Schaltplan aus \autoref{fig:schaltungohnekond} mit folgender SPICE
      directive \texttt{.dc temp 0 200 1} durchgeführt.
  }
  \label{fig:sim_dc_temp_sweep_ohne}
\end{figure}


% 1.5 Nehmen Sie die Ausgangsspannung über der Zeit für verschiedene Temperaturen in einem
% Diagramm dar und diskutieren Sie diesen Plot.
\paragraph{Temperaturvariierte Transienten-Analyse}
Für die Ausgangsspannung wurde eine temparaturvariierte Transienten-Analyse durchgeführt. Für acht verschiedene Betriebstemperaturen ist der Verlauf der Ausgangsspannung in \autoref{fig:sim_tran_temp_ohne} zu sehen.

\begin{figure}[H]
  \centering
    \includegraphics[width=\linewidth, height=7cm]{./figures/ohnekond/tempsweepausgang.png }
    \caption[Simulierte Transienten-Analyse ohne
    Überbrückungskondensator]{Simulierte Transienten-Analyse von der
      Ausgangsspannung unter Einfluss verschiedener Betriebstemperaturen. Auf
      der Abszisse befindet sich die Zeit und auf der Ordinate die
      Ausgangsspannung $U_a$. Diese Simulation wurde für den Schaltplan aus
    \autoref{fig:schaltungohnekond} mit folgenden SPICE directives \texttt{.tran 0.004 0.007 0.004 0.00001}, \texttt{.param X 10mV} und \texttt{.step temp 0 200 30} durchgeführt. Die Legende
    beinhaltet die Zuordnung von Farbe zu Temperatur (in Celsius).}
  \label{fig:sim_tran_temp_ohne}
\end{figure}


% 2.1 Bauen Sie nun den Überbrückungskondensator CE ein. Verfahren Sie wie bei 1.2 bis 1.5.
\subsubsection{Schaltung mit Überbrückungskondensator} \label{sec:Schaltung mit Überbückungskondensator}

Nun wird die Schaltung der  \autoref{fig:schaltungohnekond} um den
Überbrückungskondensator $C_E$ gemäß  \autoref{fig:schaltungmitkond}
erweitert. Weiters werden die gleichen Schritte wie beim vorigen Aufbau
durchlaufen.

\begin{figure}[H]
    \centering
    \includegraphics[width=\textwidth, height=6cm,keepaspectratio]{./figures/mitkond/messwertemitueberbrueckung.png}
    \caption{Schaltung mit Überbrückungskondensator bei einem Sinus-Eingangssignal
    mit einer Amplitude von \SI{6}{\milli\volt}, einer Frequenz von \SI{1}{\kilo\hertz} und einem
    Innenwiderstand von \SI{600}{\ohm} ; mit der Eingangs- $U_e$ und
    Ausgangsspannung $U_a$, den Eingangs- $Ce$ und Ausgangskondensatoren $Ca$, den
    Vorwiderständen $R1$ und $R2$, dem Kollektorwiderstand $RC$, dem
    Emitterwiderstand $RE$, dem Lastwiderstand $RL$, dem Basispotential $Vb$, dem
    Kollektorpotential $Vc$, dem Transistor \textit{BC107B} und der
    Betriebsspannung $UB$. Genauere Spezifikationen können dem Schaltbild entnommen
    werden.}
    \label{fig:schaltungmitkond}
\end{figure}

% FIXME(MAX) 2.2 Stellen Sie eine Eingangs-Sinusspannung von 1 kHz mit einer Amplitude innerhalb der
% FIXME(MAX) Übersteuerungstoleranzen ein, erzeugen Sie ein Simulation Profil (Time Domain) und nehmen Sie
% FIXME(MAX) jeweils die Eingangsspannung und die Ausgangsspannung in einem Plot über die Zeit auf. Lassen Sie
% FIXME(MAX) sich auch die Spannungen und Ströme im Schaltbild anzeigen 
% um den Arbeitspunkt diskutieren zu
% können. Berechnen Sie daraus die simulierte Verstärkung und diskutieren Sie die beiden Diagramme
% und ihren Zusammenhang.
\paragraph{Normalbetrieb}

Die Ein- und Ausgangsspannungen der Emitterschaltung mit
Überbrückungskondensator über der Zeit sind in \autoref{fig:sim_mit_normal_eingang} und \autoref{fig:sim_mit_normal_ausgang}
ersichtlich. Zur besseren Darstellung der jeweiligen Sinusverläufe wurden diese Spannungen getrennt abgebildet; hierbei sind die unterschiedlichen Skalierungen zu beachten.

\begin{figure}[H]
    \centering
    \includegraphics[width=\linewidth, height=7cm]{./figures/mitkond/eingangssignal6mv.png}
    \caption{Zeitlicher Verlauf der Eingangsspannung $V(u_e)$ 
    der Emitterschaltung mit Überbrückungskondensator bei einer Amplitude von
  \SI{5}{mV} des Eingangssignals}
    \label{fig:sim_mit_normal_eingang}
\end{figure}

\begin{figure}[H]
    \centering
    \includegraphics[width=\linewidth, height=7cm]{./figures/mitkond/ausgangssignal6mv.png}
    \caption{Zeitlicher Verlauf der Ausgangsspannung $V(u_a)$
    der Emitterschaltung mit Überbrückungskondensator bei einer Amplitude von
  \SI{5}{mV} des Eingangssignals}
    \label{fig:sim_mit_normal_ausgang}
\end{figure}

% FIXME(MAX) 2.3 Testen Sie wie hoch die maximale Eingangspannung werden darf bis der Transistor in der Simulation
% FIXME(MAX) übersteuert. Übersteuern Sie ihn anschließend und nehmen Sie wieder die Eingangspannung sowie
% FIXME(MAX) die Ausgangsspannung nach der Zeit auf 
% und diskutieren Sie anhand dieses Plots die auftretenden
% Verzerrungen.

\paragraph{Übersteuerungsbetrieb}

Um die Übersteuerungsgrenze zu finden, wurde ein Parameter-Sweep durchgeführt,
wobei die Amplitude der Eingangsspannung variiert wurde. In 
\autoref{fig:sim_mit_paramsweep_ausgang} ist der Verlauf der Ausgangsspannung
in Abhängigkeit der Eingangsspannung dargestellt. Dadurch konnte die
Übersteuerungsgrenze visuell bestimmt werden.

\begin{figure}[H]
    \centering
   \includegraphics[width=\linewidth, height=7cm]{./figures/mitkond/parametersweepuebersteurung.png}
    \caption{Zeitlicher Verlauf der Ausgangsspannung $V(u_a)$
    der Emitterschaltung mit Überbrückungskondensator mit einem Parameter Sweep der Amplitude des Eingangssignals.
    Folgende SPICE directives wurden für die Simulation verwendet
    \texttt{.tran 0.004 0.007 0.004 0.00001} und \texttt{.step param X 0mV 350mV 20mV}}
    \label{fig:sim_mit_paramsweep_ausgang}
\end{figure}



% FIXME(MAX) 2.4 Erstellen Sie einen DC Sweep in Abhängigkeit der Temperatur und zeigen Sie die Änderung des
% FIXME(MAX) Kollektorpotentials. 
% Diskutieren Sie die Konsequenzen einer Temperaturerhöhung.
\paragraph{DC Temperatur Sweep}
Zur Darstellung der Temperaturabhängigkeit wurde ein DC-Sweep durchgeführt. In \autoref{fig:sim_dc_temp_sweep_mit} ist das Kollektorpotential in Abhängigkeit der Temperatur zu sehen.
\begin{figure}[H]
  \centering
    \includegraphics[width=\linewidth, height=7cm]{./figures/mitkond/tempmitkondkollektorpot.png}
    \caption[Simulierter DC Temperatur Sweep mit
    Überbrückungskondensator]{Simulierter DC Temperatur Sweep, welcher die
      Temperaturabhängigkeit des Kollektorpotentials darstellt. Auf der
      Abszisse befindet sich die Temperatur des Transistors und auf der Ordinate
      das Kollektorpotential $VC$. Diese Simulation wurde für den
      Schaltplan aus \autoref{fig:schaltungmitkond} mit folgender SPICE
      directive \texttt{.dc temp 0 200 1} durchgeführt.
  }
  \label{fig:sim_dc_temp_sweep_mit}
\end{figure}

% 2.5 Nehmen Sie die Ausgangsspannung über der Zeit für verschiedene Temperaturen in einem
% Diagramm dar und diskutieren Sie diesen Plot.
\paragraph{Temperaturvariierte Transienten-Analyse}
Für die Ausgangsspannung wurde eine temparaturvariierte Transienten-Analyse durchgeführt. Für acht verschiedene Betriebstemperaturen ist der Verlauf der Ausgangsspannung in \autoref{fig:sim_tran_temp_mit} zu sehen.
\begin{figure}[H]
  \centering
    \includegraphics[width=\linewidth, height=7cm]{./figures/mitkond/ausgangmitkondtempsweep20mv.png }
    \caption[Simulierte Transienten-Analyse mit
    Überbrückungskondensator]{Simulierte Transienten-Analyse von der
      Ausgangsspannung unter Einfluss verschiedener Betriebstemperaturen. Auf
      der Abszisse befindet sich die Zeit und auf der Ordinate die
      Ausgangsspannung $U_a$. Diese Simulation wurde für den Schaltplan aus
    \autoref{fig:schaltungmitkond} mit folgenden SPICE directives \texttt{.tran
      0 0.005 0 0.0001} und \texttt{.step temp 0 200 50} durchgeführt. Die Legende
    beinhaltet die Zuordnung von Farbe zu Temperatur (in Celsius).}
  \label{fig:sim_tran_temp_mit}
\end{figure}



% 3.1 Auch wenn die Schaltung prinzipiell nicht dafür ausgerichtet ist − bauen Sie RE und CE aus und
% untersuchen Sie die Verstärkung und die Temperaturabhängigkeit des Kollektorpotentials ohne
% jegliche Rückkopplung. (Beachten Sie, dass der Arbeitspunkt dabei unvorteilhaft verschoben wird.)
\subsubsection{Schaltung ohne Emitterwiderstand \& Überbrückungskondensator}
Am Ende wird die Emitterschaltung exklusive dem Emitterwiderstand und wiederum
ohne den Überbrückungskondensator verwendet, um die starke
Temperaturabhängigkeit einer Transistorschaltung darzustellen. Die Schaltung
ist  \autoref{fig:schatlungohnekundre} zu entnehmen. Um den ursprünglichen Arbeitspunkt (Kollektorpotential von ca. \SI{7,5}{\volt}) zu erhalten, wurde dafür der Vorwiderstand $R1$ passend verringert.

\begin{figure}[H]
    \centering
    \includegraphics[width=\textwidth, height=6cm,keepaspectratio]{./figures/ohnekondundre/schaltungohnere.png}
    \caption{Schaltung ohne Emitterwiderstand und Überbrückungskondensator bei einem Sinus-Eingangssignal
      mit einer Amplitude von \SI{50}{\milli\volt}, einer Frequenz von \SI{1}{\kilo\hertz} und einem
      Innenwiderstand von \SI{600}{\ohm} ; mit der Eingangs- $U_e$ und
      Ausgangsspannung $U_a$, den Eingangs- $Ce$ und Ausgangskondensatoren $Ca$, den
      Vorwiderständen $R1$ und $R2$, dem Kollektorwiderstand $RC$, dem
      Emitterwiderstand $RE$, dem Lastwiderstand $RL$, dem Basispotential $Vb$, dem
      Kollektorpotential $Vc$, dem Transistor \textit{BC107B} und der
      Betriebsspannung $UB$. Genauere Spezifikationen können dem Schaltbild entnommen
      werden.}
    \label{fig:schatlungohnekundre}
\end{figure}

Die Spannungsverläufe für die Schaltung ohne Emitterwiderstand und
Überbrückungskondensator sind in \autoref{fig:verlaufohnekondundre} und \autoref{fig:sim_ohne_re_normal_ausgang} zu
sehen.

\begin{figure}[H]
    \centering
    \includegraphics[width=\linewidth, height=7cm]{./figures/ohnekondundre/eingangssignal10mv.png}
    \caption{Zeitlicher Verlauf der Eingangsspannung $V(u_e)$ 
    der Emitterschaltung ohne $RE$ und $CE$ bei einer Amplitude von
  \SI{20}{mV} des Eingangssignals}
    \label{fig:verlaufohnekondundre}
\end{figure}

\begin{figure}[H]
    \centering
    \includegraphics[width=\linewidth, height=7cm]{./figures/ohnekondundre/ausgangsignal10mv.png}
    \caption{Zeitlicher Verlauf der Ausgangsspannung $V(u_a)$
    der Emitterschaltung ohne $RE$ und $CE$ bei einer Amplitude von
  \SI{20}{mV} des Eingangssignals}
    \label{fig:sim_ohne_re_normal_ausgang}
\end{figure}

Damit die Verstärkung besser untersucht werden kann, wurde ein Parameter
Sweep der Eingangsamplitude durchgeführt. Die Resultate sind in folgender \autoref{fig:verst_analy} ersichtlich:

\begin{figure}[H]
    \centering
   \includegraphics[width=\linewidth, height=7cm]{./figures/ohnekondundre/verstaerkungsanalysis.png}
    \caption{Zeitlicher Verlauf der Ausgangsspannung $V(u_a)$
    der Emitterschaltung ohne $RE$ und $CE$ mit einem Parameter Sweep der Amplitude des Eingangssignals.
    Folgende SPICE directives wurden für die Simulation verwendet
    \texttt{.tran 0.004 0.007 0.004 0.00001} und \texttt{.step param X 0mV 100mV 10mV}. Sowie die obere und untere Hälften Verstärkungen $V1$ $V2$ als auch die Eingangsamplituden sind in der Grafik ersichtlich.}
    \label{fig:verst_analy}
\end{figure}


\paragraph{DC Temperatur Sweep}
Zur Darstellung der Temperaturabhängigkeit wurde ein DC-Sweep durchgeführt. In \autoref{fig:sim_dc_temp_sweep_ohne_ohne_re} ist das Kollektorpotential in Abhängigkeit der Temperatur zu sehen.
\begin{figure}[H]
  \centering
    \includegraphics[width=\linewidth, height=7cm]{./figures/ohnekondundre/dcsweepkollektorpotR2auf10kohm5mv.png }
    \caption[Simulierter DC Temperatur Sweep ohne
    Überbrückungskondensator und ohne Emitterwiderstand]{Simulierter DC Temperatur Sweep welcher die
      Temperaturabhängigkeit des Kollektorpotentials darstellt. Auf der
      Abszisse befindet sich die Temperatur des Transistors und auf der Ordinate
      das Kollektorpotential $VC$. Diese Simulation wurde bei dem
      Schaltplan aus \autoref{fig:schatlungohnekundre} mit folgender SPICE
      directive \texttt{.dc temp 0 200 1} durchgeführt.
  }
  \label{fig:sim_dc_temp_sweep_ohne_ohne_re}
\end{figure}

\paragraph{Temperaturvariierte Transienten-Analyse}
Für die Ausgangsspannung wurde eine temparaturvariierte Transienten-Analyse durchgeführt. Für elf verschiedene Betriebstemperaturen ist der Verlauf der Ausgangsspannung in \autoref{fig:sim_tran_temp_ohne} zu sehen.
\begin{figure}[H]
  \centering
    \includegraphics[width=\linewidth, height=7cm]{./figures/ohnekondundre/image.png }
    \caption[Simulierte Transienten-Analyse mit
    Überbrückungskondensator]{Simulierte Transienten-Analyse von der
      Ausgangsspannung unter Einfluss verschiedener Betriebstemperaturen. Auf
      der Abszisse befindet sich die Zeit und auf der Ordinate die
      Ausgangsspannung $U_a$. Diese Simulation wurde für den Schaltplan aus
    \autoref{fig:schatlungohnekundre} mit folgenden SPICE directives \texttt{.tran 0.004 0.007 0.004 0.00001} und \texttt{.step temp 0 100 10} durchgeführt. Die Legende
    beinhaltet die Zuordnung von Farbe zu Temperatur (in Celsius).}
  \label{fig:sim_tran_temp_ohne_ohne_re}
\end{figure}

% 1.1 Bauen Sie die Schaltung am Steckboard gemäß der berechneten Parameter ohne den Kondensator CE
% auf. Nähern Sie sich bei Ihren berechneten Widerstandswerten durch Serienschaltung der
% vorhandenen Fixwiderstände der Reihe E12 auf ein vernünftiges Maß an. Führen sie den
% Eingangsspannungsteiler mit einem Fixwiderstand und einem Potentiometer aus um den
% Arbeitspunkt anpassen zu können. Notieren Sie die tatsächlich verwendeten Widerstandswerte
% (messen Sie hierfür die Widerstände aus, diese weichen nämlich teilweise erheblich von ihrem
% nominellen Wert ab).
\subsection{Steckbrett}
Für den praktischen Teil an der Steckplatine wurden Widerstände der E12-Reihe, mit denen
die in der Vorbereitung angegebenen respektive errechneten Werte angenähert wurden,
verwendet. Zusätzlich wird für die Vorwiderstände im Spannungsteiler ein
seriell geschaltetes Potentiometer verwendet, mit welchem man sich an den
berechneten Arbeitspunkt (bei einem Kollektorpotential von ca. \SI{7,5}{\volt}) für die Emitterschaltung annähert. 

Die verwendeten Geräte sind \autoref{tab:geraeteliste} zu entnehmen.

\begin{table}
  \caption{Tabelle der verwendeten Geräte}
  \label{tab:geraeteliste}
  \centering
  \begin{tabular}{l|l}
    \hline
   \multicolumn{2}{ c }{\textbf{Geräteliste}} \\
    \hline
    \textbf{Gerät/Bauelement} & \textbf{Typ} \\
    \hline
    Oszilloskop & \textit{Tektronix TDS 2002}\\
    Funktionsgenerator & \textit{FG250D} \\
    Netzgerät & nicht bestimmbar\\
    Multimeter & \textit{Fluke 175 TrueRMS}\\
    npn-Transistor & \textit{BC 107B}\\
    Widerstände & siehe \autoref{tab:messung_widerstaende} \\
    Kondensator $C_e$ & \SI{330(40)}{\nano\farad} \\
    Kondensator $C_a$ & \SI{680(70)}{\nano\farad}\\
    Kondensator $C_E$ & \SI{270(30)}{\micro\farad}\\
    \hline
  \end{tabular}
\end{table}

Zunächst wurden die vorhandenen Widerstände gemessen, um 
diverse Abweichungen vom Nennwert festzuhalten. Die mit Unsicherheit behafteten Messwerte
sind in \autoref{tab:messung_widerstaende} zusammengefasst.

\begin{table}[H]
  \caption{Gemessene Werte der Widerstände am Steckbrett. Diese Messungen erfolgten
  mit dem \textit{Fluke 175 TrueRMS}. \\
  $R_C \dots$  Kollektorwiderstand\\
  $R_E \dots$  Emitterwiderstand\\
  $R_L \dots$  Lastwiderstand\\
  $R_1 \dots$  Erster Spannungsteilerwiderstand\\
  $R_2 \dots$  Zweiter Spannungsteilerwiderstand
  }
  \label{tab:messung_widerstaende}
  \centering
  \begin{tabular}[c]{l|l}
    $R_C$ & \SI{1.796(18)}{\kilo\ohm} \\
    $R_E$ & \SI{47.4(7)}{\ohm} \\
    $R_L$ & \SI{2.20(3)}{\kilo\ohm} \\
    $R_1$ & \SI{176.0(17)}{\kilo\ohm} \\
    $R_2$ & \SI{11.28(12)}{\kilo\ohm}
  \end{tabular}
\end{table}



Nachdem die Schaltung, wie in  \autoref{fig:aufbau} zu sehen, mitsamt 
den genannten Geräten und Bauelementen aufgebaut wurde, wurde der Schaltaufbau von den 
Laborbetreuern überprüft und verifiziert.


\begin{figure}[H]
    \centering
    \includegraphics[width=8cm, height=8cm,keepaspectratio]{./figures/foto.png}
    \caption{Am Steckbrett aufgebaute Emitterschaltung mit Überbrückungskondensator}
    \label{fig:aufbau}
\end{figure}



% 1.2 (Überprüfen der Schaltung durch einen Betreuer bevor Inbetriebnahme!)
% Verwenden Sie das Netzgerät für eine konstante Versorgungsspannung messen Sie das
% Kollektorpotential VC,A, stellen Sie es wenn nötig mit Hilfe des Potentiometers auf den gewünschten
% Wert ein. Protokollieren Sie die angepassten Widerstandswerte und die folgenden Größen UCE, UBE,
% VC, VB, IC, IB um den Arbeitspunkt evaluieren zu können.
Anschließend wurde das Netzgerät auf eine konstante
Betriebsspannung von \SI{15}{\volt} geregelt. Mithilfe des in den Vorwiderständen
seriell verschalteten Potentiometer wurde der Widerstand so variiert, um den 
gewünschten Arbeitspunkt einzustellen. Dann wurden die Werte, die in
\autoref{tab:messugenarbeitspunkt} gelistet sind, gemessen.

\begin{table}[H]
  \caption{Tabelle mit Messwerten des Arbeitspunkt. Diese Messungen wurden
  mit dem \textit{Fluke 175 TrueRMS} durchgeführt. \\
  $U_{CE} \dots$ Kollektoremitterspannung \\
  $U_{BE} \dots$ Basisemitterspannung \\
  $V_C \dots$  Kollektorpotential \\ 
  $V_B \dots$  Basispotential \\ 
  $U_{RC} \dots$ Spannungsabfall über Kollektorwiderstand  \\ 
  $U_{R1} \dots$ Spannungsabfall über $R_1$ \\
  $U_{R2} \dots$ Spannungsabfall über $R_2$
}
  \label{tab:messugenarbeitspunkt}
  \centering
  \begin{tabular}[c]{l|l}
    $U_{CE}$ &  \SI{7.22(18)}{\volt} \\
    $U_{BE}$ &  \SI{645(16)}{\milli\volt} \\
    $V_C$    &  \SI{7.48(18)}{\volt} \\
    $V_B$    &  \SI{900(30)}{\milli\volt} \\
    $U_{RC}$ &  \SI{7.70(19)}{\volt} \\
    $U_{R1}$ &  \SI{14.1(4)}{\volt} \\
    $U_{R2}$ &  \SI{840(20)}{\milli\volt}
  \end{tabular}
\end{table}


% 1.3 (Überprüfen der Schaltung durch einen Betreuer bevor Inbetriebnahme!)
% Stellen Sie nun eine sinusförmige Wechselspannung mit Hilfe des Funktionsgenerators von 1kHz
% ein und benützen Sie dieses Signal als Eingangssignal für Ihre Schaltung. Stellen Sie nun die
% Eingangspannung und die Ausgangsspannung mittels Oszilloskop dar und zwar jeweils mit und
% ohne CE. Laden Sie die Bilder oder wahlweise die Daten mittels dem Programm Open Choice
% Desktop herunter, diskutieren Sie diese, berechnen Sie die Verstärkungen und vergleichen Sie die
% Daten mit der Simulation.

\subsubsection{Normalbetrieb}
Nachdem der Aufbau erneut von den Laborbetreuenden verifiziert wurde, wurde am
Funktionsgenerator (\textit{FG250D}) das Eingangssignal auf eine Frequenz von 
\SI{999,970}{\hertz} und eine Amplitude von \SI{12,64}{\milli\volt}, sodass 
keine Übersteuerung auftrat, eingestellt.
Die Spannungsverläufe am Ein- und Ausgang wurden mittels einem 
Oszilloskop (\textit{Tektronix TDS 2002}) dargestellt und die resultierenden Oszillogramme sowie 
Daten für die nachfolgende Auswertung exportiert. Dies wurde für die Emitterschaltung im
Normalbetrieb sowohl ohne als auch mit Überbrückungskondensator durchgeführt.
Die Oszillogramme der Spannungsverläufe sind in \autoref{fig:oszi_ohne_normal} für die Schaltung
ohne und in \autoref{fig:oszi_mit_normal} für jene mit Überbrückungskondensator dargestellt. 
Um eine hohe Genauigkeit und eine leichte Ablesbarkeit der Amplituden zu erreichen, wurden die
Spannungsverläufe so skaliert, dass diese sich über den maximalen Bildbereich des Oszilloskops
erstrecken. Durch Mittelung wurde das auftretende Rauschen an Widerständen und insbesondere an den 
Halbleiterübergängen, das aufgrund der diskret fließenden Ladunsgträger auftritt, vermindert.

\paragraph{Schaltung ohne Überbrückungskondensator}
\begin{figure}[H]
  \centering
    \includegraphics[width=\linewidth, height=10cm]{./figures/messungen/ohnekond24mv.png}
  \caption{Oszillogramm der Ein- (am Kanal 2 in blau) und Augangangsspannung (am Kanal 1 in gelb) für die Schaltung auf der Steckplatine ohne Überbrückungskondensator bei einer Amplitude der Eingangsspannung von \SI{12,64}{\milli\volt}}
  \label{fig:oszi_ohne_normal}
\end{figure}

\paragraph{Schaltung mit Überbrückungskondensator}
\begin{figure}[H]
  \centering
    \includegraphics[width=\linewidth, height=10cm]{./figures/messungen/mitkond24mv.png}
  \caption{Oszillogramm der Ein- (am Kanal 2 in blau) und Augangangsspannung (am Kanal 1 in gelb) für die Schaltung auf der Steckplatine mit Überbrückungskondensator bei einer Amplitude der Eingangsspannung von \SI{12,64}{\milli\volt}}
  \label{fig:oszi_mit_normal}
\end{figure}


% 1.4 Überprüfen Sie die Übersteuerungsgrenze mit und ohne CE und nehmen Sie auch hier die
% Eingangspannung und die Ausgangsspannung im übersteuerten Betrieb mittels Oszilloskop auf.

\subsubsection{Übersteuerungsbetrieb}
Zur Darstellung des Übersteuerungsbetriebs wurde die Eingangsspannung erhöht bis die 
Übersteuerung, durch eine sichtbare Krümmung der Amplituden der Ausgangsspannung, erreicht
wurde. Dies wurde erneut für beide Schaltungen, also für jene mit und ohne 
Überbrückungskondensator, über das Oszilloskop (\textit{Tektronix TDS 2002}) dargestellt.
Die Oszillogramme zur Übersteuerung sind in \autoref{fig:oszi_ohne_uebersteuerung} für die 
Schaltung ohne und in \autoref{fig:oszi_mit_uerbersteuerung} für jene mit 
Überbrückungskondensator ersichtlich.

\paragraph{Schaltung ohne Überbrückungskondensator}
\begin{figure}[H]
  \centering
    \includegraphics[width=\linewidth, height=10cm]{./figures/messungen/ohneuebergrenze.png}
  \caption{Oszillogramm der Ein- (am Kanal 2 in blau) und Augangangsspannung (am Kanal 1 in gelb) für die Schaltung auf der Steckplatine ohne Überbrückungskondensator im Übersteuerungsbetrieb}
  \label{fig:oszi_ohne_uebersteuerung}
\end{figure}


\paragraph{Schaltung mit Überbrückungskondensator}

\begin{figure}[H]
  \centering
    \includegraphics[width=\linewidth, height=10cm]{./figures/messungen/mituebergrenze.png}
  \caption{Oszillogramm der Ein- (am Kanal 2 in blau) und Augangangsspannung (am Kanal 1 in gelb) für die Schaltung auf der Steckplatine mit Überbrückungskondensator im Übersteuerungsbetrieb}
  \label{fig:oszi_mit_uerbersteuerung}
\end{figure}

% 1.5 Variieren Sie nun die Frequenz und zeigen Sie die diesbezüglichen Grenzen der Schaltung.
% Diskutieren Sie mit Hilfe eines Oszilloskopbildes die Konsequenzen.
\subsubsection{Frequenzvariation}

Nun wurde die Frequenz des Eingangssignals variiert um die untere beziehungsweise
obere Grenzfrequenz zu ermitteln und das Verhalten der Schaltung bei
Frequenzvariation zu untersuchen. Die untere Grenzfrequenz wurde somit bestimmt und ist in
\autoref{fig:oszi_untergrenzfrequenz} ersichtlich. Die Signale sind hierbei annähernd um \SI{45}{\degree} phasenverschoben.
Die obere Grenzfrequenz wurde mit dem Frequenzgenerator nicht erreicht.

\begin{figure}[H]
  \centering
    \includegraphics[width=\linewidth, height=10cm]{./figures/messungen/unteregrenze.png}
    \caption{Diese Abbildung zeigt den zeitlichen Verlauf der Eingangs- (am Kanal 2 in blau) und
    Ausgangsspannung (am Kanal 1 in gelb) beim Betrieb an der unteren Grenzfrequenz (siehe Abbildung:
  \SI{44.9875}{\hertz})}
  \label{fig:oszi_untergrenzfrequenz}
\end{figure}


% zu 4: Auswertung siehe EPM Skript nur Besprechung von Umformungen und 
% Sachen die man mit den Messungen machen muss damit man Conclusion und Wissen 
% gewinnen kann.
% Entsprechend der in Punkt 2. angegebenen Beziehungen (Formeln) ist aus
% den Messergebnissen in Punkt 5. das in Punkt 1. formulierte Endergebnis zu berechnen.
% Oft ist eine Ermittlung des Endergebnisses aus einer grafischen Darstellung bzw. eine grafi-
% sche Veranschaulichung zweckm ̈aßig. Dabei kann die Verwendung von Millimeterpapier oder
% Computerprogrammen hilfreich sein. Wenn eine Bearbeitung der Daten auf dem Computer
% erfolgt, sollte bei der Darstellung der Graphen eine sinnvolle Skalenteilung des Koordina-
% tensystems gemacht werden. Die Unsicherheitsbetrachtung f ̈ur die angegebenen Messwerte,
% sowie f ̈ur Zwischen- und Endergebnisse ist in diesem Abschnitt nachvollziehbar zu beschrei-
% ben. Dabei ist nach Kapitel 1 vorzugehen und insbesondere auf die Klassifizierung der
% Unsicherheit (Typ-A/B) und die Unsicherheitsfortpflanzung einzugehen.
\section{Auswertung}\label{sec:Auswertung}

\subsection{Simulation}

% 1.1 Start Ausarbeitung
\subsubsection{Schaltung ohne Überbrückungskondensator}


% 1.2 Stellen Sie eine Eingangs-Sinusspannung von 1 kHz mit einer Amplitude innerhalb der
% Übersteuerungstoleranzen ein, erzeugen Sie ein Simulation Profil (Time Domain) und nehmen Sie
% jeweils die Eingangsspannung und die Ausgangsspannung in einem Plot über die Zeit auf. Lassen Sie
% sich auch die Spannungen und Ströme im Schaltbild anzeigen um den Arbeitspunkt diskutieren zu
% können. Berechnen Sie daraus die simulierte Verstärkung und diskutieren Sie die beiden Diagramme
% und ihren Zusammenhang.
\paragraph{Verstärkung}

Es wurden aus den Simulationsdaten von \textit{LTSPICE}, siehe
\autoref{fig:verlaufohnekond}, die Spitzenspannungen für $U_e$ und $U_a$
bestimmt, wodurch mit \autoref{eq:verst} die Verstärkung $V_{u}'$ ohne Überbrückungskondensator berechnet wurde:

\begin{equation}
  V_{u}' = 19
  \label{eq:sim_verst_ohne}
\end{equation}

% TODO 3D plot und erklaerung wie dieser gemacht wurde

% 1.3 Testen Sie wie hoch die maximale Eingangspannung werden darf bis der Transistor in der Simulation
% übersteuert. Übersteuern Sie ihn anschließend und nehmen Sie wieder die Eingangspannung sowie
% die Ausgangsspannung nach der Zeit auf und diskutieren Sie anhand dieses Plots die auftretenden
% Verzerrungen.

% 1.4 Erstellen Sie einen DC Sweep in Abhängigkeit der Temperatur und zeigen Sie die Änderung des
% Kollektorpotentials. Diskutieren Sie die Konsequenzen einer Temperaturerhöhung.

% 1.5 Nehmen Sie die Ausgangsspannung über der Zeit für verschiedene Temperaturen in einem
% Diagramm dar und diskutieren Sie diesen Plot.

% 2.1 Start Ausarbeitung
\subsubsection{Schaltung mit Überbrückungskondensator}


% 2.2 Stellen Sie eine Eingangs-Sinusspannung von 1 kHz mit einer Amplitude innerhalb der
% Übersteuerungstoleranzen ein, erzeugen Sie ein Simulation Profil (Time Domain) und nehmen Sie
% jeweils die Eingangsspannung und die Ausgangsspannung in einem Plot über die Zeit auf. Lassen Sie
% sich auch die Spannungen und Ströme im Schaltbild anzeigen um den Arbeitspunkt diskutieren zu
% können. Berechnen Sie daraus die simulierte Verstärkung und diskutieren Sie die beiden Diagramme
% und ihren Zusammenhang.
\paragraph{Verstärkung}

Es wurden aus den Simulationsdaten von \textit{LTSPICE}, siehe
\autoref{fig:sim_mit_normal_eingang} und \autoref{fig:sim_mit_normal_ausgang},
die Spitzenspannungen für $U_e$ und $U_a$ bestimmt, wodurch mit
\autoref{eq:verst} die Verstärkung mit Überbrückungskondensator $V_{u,C_{E}}'$
berechnet wurde:

\begin{equation}
  V_{u,C_{E}}' = 125
  \label{eq:sim_verst_mit}
\end{equation}

% TODO 3D plot und erklaerung wie dieser gemacht wurde
% 2.3 Testen Sie wie hoch die maximale Eingangspannung werden darf bis der Transistor in der Simulation
% übersteuert. Übersteuern Sie ihn anschließend und nehmen Sie wieder die Eingangspannung sowie
% die Ausgangsspannung nach der Zeit auf und diskutieren Sie anhand dieses Plots die auftretenden
% Verzerrungen.

% 2.4 Erstellen Sie einen DC Sweep in Abhängigkeit der Temperatur und zeigen Sie die Änderung des
% Kollektorpotentials. Diskutieren Sie die Konsequenzen einer Temperaturerhöhung.

% 2.5 Nehmen Sie die Ausgangsspannung über der Zeit für verschiedene Temperaturen in einem
% Diagramm dar und diskutieren Sie diesen Plot.


% 3.1 Auch wenn die Schaltung prinzipiell nicht dafür ausgerichtet ist − bauen Sie RE und CE aus und
% untersuchen Sie die Verstärkung und die Temperaturabhängigkeit des Kollektorpotentials ohne
% jegliche Rückkopplung. (Beachten Sie, dass der Arbeitspunkt dabei unvorteilhaft verschoben wird.)
\subsubsection{Schaltung ohne Emitterwiderstand \& Überbückungskondensator}

Es wurden aus den Simulationsdaten von \textit{LTSPICE}, siehe
\autoref{fig:verlaufohnekondundre} die Spitzenspannungen für $U_e$ und $U_a$
bestimmt, wodurch mit \autoref{eq:verst} die Verstärkung ohne Emitterwiderstand
und ohne Überbrückungskondensator $V_{u,\neg R_{E}}'$ berechnet wurde:

\begin{equation}
  V_{u,\neg R_{E}}' = 153
  \label{eq:sim_verst_ohne_ohne_re}
\end{equation}

\subsubsection{Verstärkung} \label{sec:ausverst}
Um die Verstärkung zu bestimmen, wurde über die Amplituden des zeitlichen
Verlaufs der Ausgangs- und Eingangsspannungen gemittelt und deren Verhältnis
gemäß \autoref{eq:verst} berechnet. Somit ergeben sich die Werte der
\autoref{tab:verst_sim_alle} für die Verstärkungen der drei Schaltungen aus dem Abschnitt
\nameref{sec:Versuchsim}. 


\begin{table}[H]
  \caption{Tabelle mit allen ermittelten Verstärkungen \\
  $V_{u}' \dots$ Verstärkung ohne Überbrückungskondensator \\
  $V_{u,C_{E}}' \dots$   Verstärkung mit Überbrückungskondensator \\
  $V_{u,\neg R_{E}}' \dots$ Verstärkung ohne Emitterwiderstand und ohne Überbrückungskondensator 
  }
  \label{tab:verst_sim_alle}
  \centering
  \begin{tabular}{l|l}
  $V_{u}'$            & 19 \\
  $V_{u,C_{E}}'$      & 125 \\
  $V_{u,\neg R_{E}}'$ & 153
  \end{tabular}
\end{table}



% 1.1 Bauen Sie die Schaltung am Steckboard gemäß der berechneten Parameter ohne den Kondensator CE
% auf. Nähern Sie sich bei Ihren berechneten Widerstandswerten durch Serienschaltung der
% vorhandenen Fixwiderstände der Reihe E12 auf ein vernünftiges Maß an. Führen sie den
% Eingangsspannungsteiler mit einem Fixwiderstand und einem Potentiometer aus um den
% Arbeitspunkt anpassen zu können. Notieren Sie die tatsächlich verwendeten Widerstandswerte
% (messen Sie hierfür die Widerstände aus, diese weichen nämlich teilweise erheblich von ihrem
% nominellen Wert ab).
\subsection{Steckbrett}



% 1.2 (Überprüfen der Schaltung durch einen Betreuer bevor Inbetriebnahme!)
% Verwenden Sie das Netzgerät für eine konstante Versorgungsspannung messen Sie das
% Kollektorpotential VC,A, stellen Sie es wenn nötig mit Hilfe des Potentiometers auf den gewünschten
% Wert ein. Protokollieren Sie die angepassten Widerstandswerte und die folgenden Größen UCE, UBE,
% VC, VB, IC, IB um den Arbeitspunkt evaluieren zu können.
Da ein Großteil, der den Arbeitspunkt definierenden Größen, direkt gemessen werden
konnten, werden hier nur noch die fehlenden Größen mit den Messwerten aus
\autoref{tab:messugenarbeitspunkt} errechnet. Es sind nur noch der Basisstrom
$I_B$ und der Kollektorstrom $I_C$ zu bestimmen. Durch Verwendung
\autoref{eq:ohm} kann $I_C$ durch den Spannungsabfall über den
Kollektorwiderstand bestimmt werden. Da $I_B$ nichts anderes ist als die
Differenz von dem Strom durch $R_1$ und dem Strom durch $R_2$ (siehe
\autoref{fig:schaltungohnekond}) kann $I_B$ äquivalent zu $I_C$ berechnet
werden. Für die Fehlerfortpflanzung ist das Gauß'sche Fehlerfortpflanzungsgesetz
verwendet worden.

\begin{table}[H]
  \caption{ Tabelle mit den fehlenden zu berechnenden Größen des Arbeitspunkts \\
  $I_C \dots$ Kollektorstrom \\
  $I_B \dots$ Basisstrom 
  }
  \label{tab:aus_arbeitspunkt_daten}
  \centering
  \begin{tabular}[c]{l|l}
    $I_C$ & \SI{4.29(12)}{\milli\ampere} \\
    $I_B$ & \SI{6(4)}{\micro\ampere} \\
  \end{tabular}
\end{table}


% 1.3 (Überprüfen der Schaltung durch einen Betreuer bevor Inbetriebnahme!)
% Stellen Sie nun eine sinusförmige Wechselspannung mit Hilfe des Funktionsgenerators von 1kHz
% ein und benützen Sie dieses Signal als Eingangssignal für Ihre Schaltung. Stellen Sie nun die
% Eingangspannung und die Ausgangsspannung mittels Oszilloskop dar und zwar jeweils mit und
% ohne CE. Laden Sie die Bilder oder wahlweise die Daten mittels dem Programm Open Choice
% Desktop herunter, diskutieren Sie diese, berechnen Sie die Verstärkungen und vergleichen Sie die
% Daten mit der Simulation.
\subsubsection{Normalbetrieb}
Die Verstärkungsfaktoren wurden analog wie für die Simulation im
vorigen Abschnitt \nameref{sec:ausverst} ermittelt.

\paragraph{Schaltung ohne Überbrückungskondensator}

Da bei der Aufnahme der Daten das Oszilloskop so eingestellt wurde, dass das
Signal die "Full-Scale" komplett ausnutzt, kann die Verstärkung $V_{u}'$ durch
Division der Volts per Division vom Eingangs- und Ausgangssignal, welche im
Oszillogramm, siehe \autoref{fig:oszi_ohne_normal}, ersichtlich sind, errechnet
werden.

\begin{equation}
  V_{u}' = 17
  \label{eq:oszi_verst_ohne}
\end{equation}

\paragraph{Schaltung mit Überbrückungskondensator}

Da bei der Aufnahme der Daten das Oszilloskop so eingestellt wurde, dass das
Signal die "Full-Scale" komplett ausnutzt, kann die Verstärkung $V_{u,C_{E}}'$
durch Division der Volts per Division vom Eingangs- und Ausgangssignal,
welche im Oszillogramm, siehe \autoref{fig:oszi_mit_normal}, ersichtlich sind,
errechnet werden.

\begin{equation}
  V_{u,C_{E}}' = 95
  \label{eq:oszi_verst_mit}
\end{equation}



% zu 5: Diskussion und Zusammenfassung
% In der Zusammenfassung stehen noch einmal die wichtigsten Messergebnisse, wobei auf Tabellen und
% Abbildungen nur verwiesen werden soll. Die Ergebnisse sind auch zu diskutieren. Insbesondere müssen
% Abweichungen zwischen Simulation und praktischer Durchführung diskutiert werden.
\section{Diskussion und Zusammenfassung}\label{sec:Diskussion} 
\subsection{Diskussion}
% 1.2 Stellen Sie eine Eingangs-Sinusspannung von 1 kHz mit einer Amplitude innerhalb der
% Übersteuerungstoleranzen ein, erzeugen Sie ein Simulation Profil (Time Domain) und nehmen Sie
% jeweils die Eingangsspannung und die Ausgangsspannung in einem Plot über die Zeit auf. Lassen Sie
% sich auch die Spannungen und Ströme im Schaltbild anzeigen um den Arbeitspunkt diskutieren zu
% können. Berechnen Sie daraus die simulierte Verstärkung und diskutieren Sie die beiden Diagramme
% und ihren Zusammenhang.
\subsubsection{Verstärkung}
Die Verstärkung der Schaltung ohne Überbrückungskondensator sollte entsprechend der
Aufgabenstellung \num{20} betragen. Die für diese Schaltung in der Simulation erzielte Verstärkung ergab 
sich zu \num{19}, am Steckbrett zu \num{17}. Somit sind nur geringe Abweichungen von \num{1} für die Simulation
respektive \num{3} für die Steckbrett-Schaltung gegenüber der anvisierten Verstärkung aufgetreten. Diese
Unterschiede könnten auf marginale Abweichungen der jeweiligen Arbeitspunkte zurückzuführen sein. Die Kollektorpotentiale
in der Simulation und an der Steckplatine weisen zum theoretischen Kollektorpotential von \SI{7,50}{\volt} Unterschiede 
im \SI{10}{\milli\volt}-Bereich auf. Da die Werte der Widerstände und Kondensatoren der Steckbrett-Schaltung nur
angenähert wurden und die Widerstände gemäß \autoref{tab:messung_widerstaende} Abweichungen vom Nennwert aufwiesen, ist
die relativ zur Simulation größere Abweichung nachvollziehbar.
\newline
Der theoretische Verstärkungsfaktor für die Emitterschaltung mit Überbrückungskondensator wurde in \autoref{sec:Vorbereitung}
zu \num{117} berechnet. In der Simulation wurde dieser zu \num{125} und am Steckbrett zu \num{95} ermittelt. Diese weisen nun
zwar größeren Differenzen vom theoretischen Wert, nämlich \num{8} beziehungsweise \num{22} auf, sind aber in einer ähnlichen 
Größenordnung wie ebenjener. Die Abweichungen der Verstärkungen und auch die relativ zur Simulation größere Abweichung 
der Steckbrett-Schaltung erklärt sich analog zu den aufgetretenen Differenzen für die Verstärkungsfaktoren der 
Emitterschaltung ohne Überbrückungskondensator. Aufgrund der nun fast 6-fach größeren Verstärkung für den Schaltaufbau
mit Überbrückungskondensator ist auch eine verhältnismäßig größere Abweichung der ermittelten Werten verständlich.
\newline
Anhand der Spannungsverläufe der \autoref{fig:verlaufohnekond}, \autoref{fig:sim_ohne_normal_ausgang}, \autoref{fig:sim_mit_normal_eingang},
\autoref{fig:sim_mit_normal_ausgang}, \autoref{fig:verlaufohnekondundre}, \autoref{fig:sim_ohne_re_normal_ausgang},
\autoref{fig:oszi_ohne_normal} und \autoref{fig:oszi_mit_normal} ist zudem sowohl für den Simulations- als auch für den Steckbrettversuch die 
charakteristische Invertierung der Emitterschaltung des Ausgangs- zum Eingangssignal zu sehen.
% 1.3 Testen Sie wie hoch die maximale Eingangspannung werden darf bis der Transistor in der Simulation
% übersteuert. Übersteuern Sie ihn anschließend und nehmen Sie wieder die Eingangspannung sowie
% die Ausgangsspannung nach der Zeit auf und diskutieren Sie anhand dieses Plots die auftretenden
% Verzerrungen.
\subsubsection{Übersteuerung}
Die Spannungsverläufe der Oszillogramme der Steckbrett-Schaltung, siehe 
\autoref{fig:oszi_ohne_uebersteuerung} und \autoref{fig:oszi_mit_uerbersteuerung}, zeigen deutlich (wie vom
Laborbetreuenden gefordert) den Übersteuerungsbetrieb der Transistorschaltung. Dabei sind die 
unterschiedlichen Breiten, Krümmungen und Größen der Amplituden der beiden Halbachsen gut sichtbar - 
besonders im Vergleich zum Normalbetrieb. Bei der Verstärkung ist dementsprechend eine Präferenz einer der beiden Halbachsen in den Diagrammen evident. Zu erkennen ist weiters, dass die Emitterschaltung mit 
Überbrückungskondensator beträchtlich sensibler ist und schon bei einer deutlich geringeren Amplitude der 
Eingangsspannung (vier Mal geringer als ohne dem Kondensator) sättigt und in den Bereich der Übersteuerung
übergeht. Das stimmt auch mit der Bedeutung und den Größenordnungen der in \autoref{sec:Vorbereitung} errechneten Maximalwerte für die Eingangsspannung, bei der noch nicht übersteuert wird, überein.
\newline
Für die Simulation wurde die Übersteuerungsgrenze, also die Sättigung, visuell durch den Parameter-Sweep in
\autoref{sec:Versuchsim} bestimmt (siehe \autoref{fig:sim_mit_paramsweep_ausgang}, \autoref{fig:sim_ohne_paramsweep_ausgang}). Das Ergebnis deckt sich mit der vorhergehenden Beobachtung anhand der 
Oszillogramme in \autoref{fig:oszi_ohne_uebersteuerung} und \autoref{fig:oszi_mit_uerbersteuerung}, dass 
die Schaltung mit Überbrückungskondensator empfindlicher auf eine größere Amplitude der Eingangsspannung 
reagiert und demzufolge früher die Übersteuerungsgrenze erreicht als jene Schaltung ohne 
Überbrückungskondensator.

\subsubsection{Temperaturabhängigkeit}
\paragraph{Kollektorpotential}
% 1.4 Erstellen Sie einen DC Sweep in Abhängigkeit der Temperatur und zeigen Sie die Änderung des
% Kollektorpotentials. Diskutieren Sie die Konsequenzen einer Temperaturerhöhung.
Die Konsequenzen einer stetigen Temperaturerhöhung sind anhand des Kollektorpotentials in
\autoref{fig:sim_dc_temp_sweep_ohne}, \autoref{fig:sim_dc_temp_sweep_mit} und \autoref{fig:sim_dc_temp_sweep_ohne_ohne_re}
ersichtlich. Dabei wird die Bedeutung des Emitterwiderstands, der in \autoref{fig:sim_dc_temp_sweep_ohne_ohne_re}
im Gegensatz zu den anderen beiden Abbildungen nicht eingebaut wurde, klar. Die Diagramme zu den beiden Schaltungen mit 
Emitterwiderstand sind annähernd identisch und fallen zwar monoton ab, aber nur mit einer Steigung von ungefähr \SI{-53}{\milli\volt\per\kelvin}. Dahingegen 
ist die Abnahme für die Schaltung ohne Emitterwiderstand viel steiler und beträgt \SI{-120}{\milli\volt\per\kelvin}. Diese Steigung resultiert daraus, dass 
durch die Zunahme der Temperatur die Leitfähigkeit des Halbleiters zunimmt. Dadurch steigt weiters der Strom $I_C$ und es fällt am Kollektorwiderstand eine höhere
Spannung ab, woraus ein geringeres Kollektorpotential resultiert. Aufgrund des starken Abfalls des Kollektorpotentials ohne Emitterwiderstand ist die Schaltung
für den errechneten Arbeitspunkt nur in einem sehr kleinen Temperaturintervall verwendbar. Der Emitterwiderstand  kann allerdings für die anderen beiden Schaltungen
(siehe \autoref{fig:sim_dc_temp_sweep_ohne} und \autoref{fig:sim_dc_temp_sweep_mit}) mittels Gegenstromkopplung die Temperaturabhängigkeit reduzieren. Erst ab einer Temperatur
von ungefähr \SI{50}{\celsius} ist das Kollektorpotential zu weit vom gewollten Kollektorpotential für den Arbeitspunkt entfernt, um die anvisierten Verstärkungen zu
erhalten. Da Transistoren im Allgemeinen (außerhalb des Labors) bei Temperaturen über \SI{50}{\celsius} betrieben werden, müsste noch eine zusätzliche
Kühlung verwendet werden, um den Arbeitspunkt adäquat zu gewährleisten.
\newline
Zu beachten ist, dass in der Simulation die Temperaturabhängigkeit von Widerständen und Kondensatoren nicht berücksichtigt wird, was in einer realen Schaltung
zu Abweichungen von der Simulation führen könnte, da der Widerstand mit steigender Temperaturen zu- und die Kapazität abnimmt.

% 1.5 Nehmen Sie die Ausgangsspannung über der Zeit für verschiedene Temperaturen in einem
% Diagramm dar und diskutieren Sie diesen Plot.
\paragraph{Ausgangsspannung}
Ein Vergleich der temperaturvariierten Transienten-Analysen für die jeweiligen Ausgangsspannungen, die in \autoref{fig:sim_tran_temp_ohne},
\autoref{fig:sim_tran_temp_mit} und \autoref{fig:sim_tran_temp_ohne_ohne_re} dargestellt werden, zeigen wiederum eine besonders starke Temperaturabhängigkeit 
für die Schaltung ohne Emitterwiderstand (\autoref{fig:sim_tran_temp_ohne_ohne_re}). Obwohl die anderen Schaltungen genauso bei hohen Temperaturen Verzerrungen
des Sinusverlaufs zur Folge haben, ist zu beachten, dass die Analyse für die Emitterschaltung ohne Emitterwiderstand schon bei \SI{100}{\celsius} abgebrochen wurde und schon 
davor zu besagten Verzerrungen der Ausgangssignals führen. Dabei scheint wiederum eine Präferenz der positiven Halbachse
aufzutreten. Dies könnte auf die Zunahme der Leitfähigkeit des Halbleiters durch die Temperaturerhöhung zurückzuführen sein, wobei die Abbildungen der
Höchsttemperaturen sehr stark das Verhalten einer Diode widerzuspiegelen scheinen.

% 2.2 Stellen Sie eine Eingangs-Sinusspannung von 1 kHz mit einer Amplitude innerhalb der
% Übersteuerungstoleranzen ein, erzeugen Sie ein Simulation Profil (Time Domain) und nehmen Sie
% jeweils die Eingangsspannung und die Ausgangsspannung in einem Plot über die Zeit auf. Lassen Sie
% sich auch die Spannungen und Ströme im Schaltbild anzeigen um den Arbeitspunkt diskutieren zu
% können. Berechnen Sie daraus die simulierte Verstärkung und diskutieren Sie die beiden Diagramme
% und ihren Zusammenhang.

% 2.3 Testen Sie wie hoch die maximale Eingangspannung werden darf bis der Transistor in der Simulation
% übersteuert. Übersteuern Sie ihn anschließend und nehmen Sie wieder die Eingangspannung sowie
% die Ausgangsspannung nach der Zeit auf und diskutieren Sie anhand dieses Plots die auftretenden
% Verzerrungen.

% 2.4 Erstellen Sie einen DC Sweep in Abhängigkeit der Temperatur und zeigen Sie die Änderung des
% Kollektorpotentials. Diskutieren Sie die Konsequenzen einer Temperaturerhöhung.

% 2.5 Nehmen Sie die Ausgangsspannung über der Zeit für verschiedene Temperaturen in einem
% Diagramm dar und diskutieren Sie diesen Plot.

% 3.1 Auch wenn die Schaltung prinzipiell nicht dafür ausgerichtet ist − bauen Sie RE und CE aus und
% untersuchen Sie die Verstärkung und die Temperaturabhängigkeit des Kollektorpotentials ohne
% jegliche Rückkopplung. (Beachten Sie, dass der Arbeitspunkt dabei unvorteilhaft verschoben wird.)
% ERLANGTE ERKENNTNIS!
%starke Tempabh. siehe oben

% 1.3 (Überprüfen der Schaltung durch einen Betreuer bevor Inbetriebnahme!)
% Stellen Sie nun eine sinusförmige Wechselspannung mit Hilfe des Funktionsgenerators von 1kHz
% ein und benützen Sie dieses Signal als Eingangssignal für Ihre Schaltung. Stellen Sie nun die
% Eingangspannung und die Ausgangsspannung mittels Oszilloskop dar und zwar jeweils mit und
% ohne CE. Laden Sie die Bilder oder wahlweise die Daten mittels dem Programm Open Choice
% Desktop herunter, diskutieren Sie diese, berechnen Sie die Verstärkungen und vergleichen Sie die
% Daten mit der Simulation.

% 1.4 Überprüfen Sie die Übersteuerungsgrenze mit und ohne CE und nehmen Sie auch hier die
% Eingangspannung und die Ausgangsspannung im übersteuerten Betrieb mittels Oszilloskop auf.
\subsubsection{Frequenzabhängigkeit}
Die untere Grenzfrequenz, siehe Abbildung \autoref{fig:oszi_untergrenzfrequenz}, folgt aus der Verwendung von Hochpässen, 
sodass niedrige Frequenzen geschwächt werden. Diese weicht um ungefähr \SI{10}{\hertz} vom theoretisch errechneten Wert ab,
was aufgrund der Abweichungen der realen Schaltung nachvollziehbar ist. Die Größenordnungen korrespondieren allerdings.
Die obere Grenzfrequenz, die mit dem Frequenzgenerator nicht erreicht wurde, ergibt sich aus der endlichen Beweglichkeit 
der Ladungsträger.
% 1.5 Variieren Sie nun die Frequenz und zeigen Sie die diesbezüglichen Grenzen der Schaltung.
% Diskutieren Sie mit Hilfe eines Oszilloskopbildes die Konsequenzen.

% 1.6 Während Sie das Live-Oszilloskopbild betrachten, können Sie mittels Körpertemperatur das
% Metallgehäuse des Transistors erwärmen. Schauen Sie sich die Folgen genau an und diskutieren Sie
% diese.


%Man achte hierbei auf die Volts per division bei gleicher Eingangsspannung
%erzeugt die Schaltung mit Kondensator eine um fast 6-fache größere Verstärkung.

%Es wäre wohl vernünftiger gewesen, die Schaltung an der Steckplatine zunächst aufzubauen, um etwaige 
%Ungenauigkeiten durch die Widerstände 
\subsection{Zusammenfassung}
Die für die Simulation bestimmten Verstärkungen der drei Emitterschaltungen sind der \autoref{tab:verst_sim_alle} zu
entnehmen. Die reale Verstärkung der Schaltung am Steckbrett ohne Überbrückungskondensator ergibt sich zu \num{17}, für jene
mit Kondensator beträgt sie \num{95}.

\newpage

\printbibliography

\listoffigures

\listoftables



\end{document}

